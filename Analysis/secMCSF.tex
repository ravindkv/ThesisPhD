\section{Luminosity scale factors}
\label{s:lumi_sf}

The simulated events listed in Table~\ref{tab:mcSample} are arbitrarily generated.
While comparing distributions obtained using simulated events with that of data, 
the former has to be normalised with the integrated luminosity of data.
To achieve this, each simulated sample is multiplied by the following luminosity scale 
factor ($\rm{SF}_{L}$):
\begin{equation}
\rm{SF}_{L}=\frac{L_{\rm {data}}}{L_{\rm {simulation}}} = \frac{L_{\rm {data}}\times\sigma_{\rm {simulation}}}{N_{\rm {simulation}}},
\label{eq:lumiSF}
\end{equation}
where $L_{\rm data}$ = 35.9\fbinv and $N_{\rm {simulation}}$ is the number of 
simulated events and $\sigma_{\rm {simulation}}$ is the corresponding cross section for 
the given sample. As can be seen from Table~\ref{tab:mcSample}, the $\rm{SF}_{L}$ value is less 
than 1 for most of the background samples except for QCD multijet. For QCD multijet samples, 
it is very large compared to other backgrounds owing to less number of generated events compared 
to its cross section. Simulation of large number of QCD multijet events is not feasible because of 
computing limitations.  

\section{Pileup reweighting}
\label{s:pileup_reweighting}
The pileup distributions in the data and simulation are different. To have the
same pileup distribution, the simulated events are reweighted. Depending on 
the number of primary vertices, the simulated event is multiplied by corresponding 
pileup weight. The pileup weights are the ratio of pileup distributions from 
data and simulation. 

\section{Lepton scale factors}
\label{s:lepton_sf}
Trigger, tracking, isolation and identification efficiencies are different between simulation and data.
A $\pt$ and $\eta$ dependent scale factors are applied to simulated events to take care of this 
difference. The muon scale factors are era dependent. There are different muon scale factors for era 
BCDEF and GH. However, electron scale factors are the same for full 2016 data taking period.
Separate scale factors are combined into one as
\begin{equation}
 \rm {SF}^\mu = \rm {SF}^\mu_{\rm {ID}}\times \rm {SF}^\mu_{\rm {iso}}\times \rm {SF}^\mu_{\rm {track}}\times \rm {SF}^\mu_{\rm {trig}}, \quad
 \rm {SF}^{ele} = \rm {SF}^{ele}_{\rm {ID}}\times \rm {SF}^{ele}_{\rm {reco}}\times \rm {SF}^{ele}_{\rm {trig}},
\label{eq:lepSF}
\end{equation}
where $\rm {SF}^\mu_{\rm {ID}}$, $\rm {SF}^\mu_{\rm {iso}}$, $\rm {SF}^\mu_{\rm {track}}$ and $\rm {SF}^\mu_{\rm {trig}}$ are weighted
average with luminosity ($L$) for different eras (BCDEF and GH) \eg 
\begin{equation}
 \rm {SF}^\mu_{\rm {ID}}=\frac{\rm {SF}^\mu_{\rm {ID}}({\rm {BCDEF}})\times L({\rm {BCDEF}})+\rm {SF}^\mu_{\rm {ID}}({\rm {GH}})\times L({\rm {GH}})}
 {L({\rm {BCDEF}})+L({\rm {GH}})},
\end{equation}
where, for example, $\rm {SF}^\mu_{\rm {ID}}({\rm {BCDEF}})$ is the scale factor for 
era-B to era-F and L({\rm {BCDEF}}) is the luminosity during this period. A similar formula 
for $\rm {SF}^\mu_{\rm {iso}}$, $\rm {SF}^\mu_{\rm {track}}$ and $\rm {SF}^\mu_{\rm {trig}}$ is used. The scale 
factors of Equation (\ref{eq:lepSF}) are applied on simulated events. 

\section{Jet and \MET correction}
\label{s:JEC}
Jets from simulations are smeared using jet energy scale (JES) and jet energy 
resolution (JER) to have the same resolution as that in data~\cite{Khachatryan:2016kdb}.
For smearing, $\pt$ and $\eta$ dependent scale factors are used as listed 
in Table~\ref{tab:jer_sf}. The $\pt$ of jet in simulations is scaled by the 
following factor:
\begin{equation}
    {\pt}_{\text{scale}} = {\rm {max}}[0.0,1.0+(\rm{SF}-1)\times({\pt}_{\text{jet}}-{\pt}_{\text{jet}}^{\text{gen}})/{\pt}_{\text{jet}}]
\end{equation}
with the following constraint:
\begin{equation}
    {\pt}_{\text{jet}}^{\text{gen}}> 0, \quad \Delta \rm R<0.2, \quad |{\pt}_{\text{jet}}-{\pt}_{\text{jet}}^{\text{gen}} |<
    3\times\sigma_{\rm {JER}}\times{\pt}_{\text{jet}},
\end{equation}
where $\Delta \rm R$ is the angular separation between the reconstructed and 
generated jet and $\sigma_{\text{JER}}$ is resolution of \pt of reconstructed jet. 
\begin{table}
    \caption{Jet energy resolution scale factors for different $\eta$ range}
 \label{tab:jer_sf}
 \begin{center}
 \begin{tabular}{cccc}
     \hline
     \hline
     $\eta$ range & base & down & up \\ 
     \hline
     \hline
     $0.0 \leq | \eta |< 0.5 $ & 1.109 & 1.044 & 1.174 \\
     $0.5 \leq | \eta |< 0.8 $ & 1.138 & 1.072 & 1.204 \\
     $0.8 \leq | \eta |< 1.1 $ & 1.114 & 1.050 & 1.178 \\
     $1.1 \leq | \eta |< 1.3 $ & 1.123 & 1.022 & 1.224 \\
     $1.3 \leq | \eta |< 1.7 $ & 1.084 & 0.985 & 1.183 \\
     $1.7 \leq | \eta |< 1.9 $ & 1.082 & 0.973 & 1.191 \\
     $1.9 \leq | \eta |< 2.1 $ & 1.140 & 1.020 & 1.260 \\
     $2.1 \leq | \eta |< 2.3 $ & 1.067 & 0.953 & 1.181 \\
     $2.3 \leq | \eta |< 2.5 $ & 1.177 & 0.967 & 1.387 \\
     $2.5 \leq | \eta |< 2.8 $ & 1.364 & 1.203 & 1.525 \\
     $2.8 \leq | \eta |< 3.0 $ & 1.857 & 1.654 & 2.060 \\
     $3.0 \leq | \eta |< 3.2 $ & 1.328 & 1.203 & 1.453 \\
     $3.2 \leq | \eta |< 5.0 $ & 1.160 & 1.013 & 1.307 \\ \hline
 \end{tabular}
 \end{center}
 \end{table}

\section{\text{b} tag scale factor}
\label{s:bTagSF}
The \PQb tagging efficiency is different between simulation and data.
An event weight as given by Equation (\ref{eq:btagWt}) is applied on the simulated events to 
take care of this difference ~\cite{BTagSFMethods}. 
\begin{equation}
P(\text{Simulation}) = \prod_{\text{i = tagged}} \epsilon_i \prod_{\text{j = not tagged}}(i -\epsilon_j)
\end{equation}
\begin{equation}
P(\text{Data}) = \prod_{\text{i = tagged}} \text{SF}_{i}\epsilon_i \prod_{\text{j = not tagged}} (1-\text{SF}_j\epsilon_j)
\end{equation}
\begin{equation}
w = \frac{P(\text{Data})}{P(\text{Simulation})}
\label{eq:btagWt}
\end{equation}
 The \PQb tagging efficiency $\epsilon_f(m,n)$ in the $(m,n)$ bin of $\pt$ and $\eta$ is calculated 
 using the following formula                                                               
 \begin{equation}                                                                          
  \epsilon_f(m,n)=\frac{N^{\rm {\PQb-tagged}}_f(m,n)}{N^{\rm {total}}_f(m,n)},                   
 \label{eq:btag_eff}                                                                       
 \end{equation}                                                                            
 where $N^{\rm {\PQb-tagged}}_f(m,n)$ is the number of \PQb-tagged jets with flavor \text{f}
(\PQb quark, \PQc quark, light quark, and gluon) and $N^{\rm {total}}_f(m,n)$ is the total number 
of events. The scale factor ($\text{SF}_i$) depends on $\pt$, $\eta$, parton flavor of jet, and 
\PQb jet discriminator value. 

