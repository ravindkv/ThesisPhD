\section{Detectors at the LHC}
\begin{itemize}[leftmargin=*]		
	\item $\textbf{ATLAS}$ (A Toroidal LHC ApparatuS):  The 
	ATLAS~\cite{Collaboration_2008_ATLAS} is a general purpose
	detector built to study the physics of the standard model and beyond it such
	as the origin of dark matter. It was one of the two detectors that discovered the 
	Higgs Boson in 2012~\cite{Aad:2012tfa}. It is 25\unit{m} in diameter, 
	contains around 3000\unit{km} of cable, 46\unit{m} long, and weighs nearly 7000 tonnes.
\item $\textbf{LHCf}$ (Large Hadron Collider forward): The 
	LHCf~\cite{Collaboration_2008_LHCf} is placed at IR1, 
	on both sides of the ATLAS detector. It is designed to study collisions in the 
	forward region that appreciates very high radiation. 
	One of the physics goals of LHCf is to study the neutral pions ($\pi^0$) produced in 
	collisions. A Proper measurement of the energy of $\pi^0$ will help to understand and double
	check the origin of ultra-high-energy cosmic rays which have already been 
	measured by other experiments such as the Telescope Array Project in Utah,
	and the Pierre Auger Observatory in Argentina.
\item $\textbf{ALICE}$ (A Large Ion Collider Experiment): 
	The ALICE~\cite{Collaboration_2008_ALICE} detector is mainly designed
	for the study of heavy ion collisions such as proton-lead (p-Pb) and lead-lead 
	(Pb-Pb). At the collision point, due to an extremely high temperature, the 
	quark-gluon plasma is produced. It is believed that similar conditions existed 
	just after the Big Bang where quarks and gluons were in a free state before 
	combining to form hadrons. In 2011, the ALICE experiment measured the size of
	the fireball in Pb-Pb collision \cite{Aamodt:2011mr}. The ALICE with a weight 
	of 10000 tonnes, weighs more than the Eiffel tower.
\item $\textbf{CMS}$ (Compact Muon Solenoid): The CMS~\cite{Collaboration_2008_CMS} 
	is a general purpose detector 
	like the ATLAS. A detailed information about this experiment is given in
	Section~\ref{c:secCMS}.
\item $\textbf{TOTEM}$ (TOTal Elastic and diffractive cross section Measurement): The
	TOTEM~\cite{Collaboration_2008_TOTEM} is a small detector placed at IR5 in the 
	same cavern where the CMS is.
	As the name suggests, its aim is to measure the total cross-section, and to study 
	diffractive processes and elastic scattering.
\item $\textbf{LHCb}$ (Large Hadron Collider beauty): The 
	LHCb~\cite{Collaboration_2008_LHCb} experiment specially built 
	for the study of physics processes involving \PQb and \PQc quarks such as the parameters of
	CP violation from the decays of \PQb hadrons. During 2010-2012,
	LHCb has published many physics results including the measurement of branching
	fraction of the $B_{s} \rightarrow \mu^+ \mu^-$ decay~\cite{Aaij:2012nna}, the 
	forward-backward symmetry of muon pair from the $B_{d} \rightarrow K^* \mu^+ \mu^-$ decay, 
	the properties of radiative B decays, the determination of the unitarity
	triangle parameters, and two-body charmless decay of B mesons.
\item $\textbf{MoEDAL}$ (Monopole and Exotics Detector At the LHC): 
	The MoEDAL~\cite{Acharya:2014nyr} 
	detector is placed in IR8 adjacent to the LHCb. Its physics goal 
	is to search for the existence of magnetic monopoles. So far it has not found any 
	evidence for magnetic monopoles and has accordingly set an exclusion limit on 
	their production cross section.
\end{itemize}
