%\phantomsection\addcontentsline{toc}{chapter}{List of publications} \noindent
\chapter {Publications}
During the first two years of my Ph.D I worked in the area of theoretical high energy physics. And the work during this period has resulted in three publications (first three entries in the list below). Thereafter, I switched to experiments and became a member of CMS collaboration, LHC. Being a member of this collaboration, I am one of the co-author of more than 160 papers by now (August 2019). The list of all the papers can be found at~\cite{rvermaPub}. However, the papers where I have contributed directly 
(enlisted below) and the work due to which I got the authorship in the rest of the papers published 
by the CMS collaboration is described in the next point.

%\section*{Direct contribution}
\begin{itemize}[leftmargin=*]
\item {\textbf{Direct contribution:}
\begin{enumerate}[leftmargin=*]
%\cite{Agarwal:2014oaa}
\item%{Agarwal:2014oaa}
{\bf {ELKO fermions as dark matter candidates}}
  \\{}B.~Agarwal, P.~Jain, S.~Mitra, A.~C.~Nayak and R.~K.~Verma.
  \\{}arXiv:1407.0797 [hep-ph]
  \\{}DOI:10.1103/PhysRevD.92.075027
  \\{}Phys.\ Rev.\ D {\bf 92}, 075027 (2015)
%(Jul 3, 2014)
\inspireurl{http://inspirehep.net/record/1304694}
\citations{17 citations counted in INSPIRE as of 23 Apr 2019}

%\cite{Nayak:2015xba}
\item%{Nayak:2015xba}
{\bf {Effect of VSR invariant Chern-Simon Lagrangian on photon polarization}}
  \\{}A.~C.~Nayak, R.~K.~Verma and P.~Jain.
  \\{}arXiv:1504.04921 [hep-ph]
  \\{}DOI:10.1088/1475-7516/2015/07/031
  \\{}JCAP {\bf 1507}, no. 07, 031 (2015)
%(Apr 19, 2015)
\inspireurl{http://inspirehep.net/record/1362189}
\citations{6 citations counted in INSPIRE as of 23 Apr 2019}

\item%{Jain:2016kai}
{\bf {The top threshold effect in the $\gamma\gamma$ production at the LHC}}
  \\{}S.~R.~Dugad, P.~Jain, S.~Mitra, P.~Sanyal and R.~K.~Verma.
  \\{}arXiv:1605.07360 [hep-ph]
  \\{}DOI:10.1140/epjc/s10052-018-6188-z
  \\{}Eur.\ Phys.\ J.\ C {\bf 78}, no. 9, 715 (2018)
%(May 24, 2016)
\inspireurl{http://inspirehep.net/record/1465525}
\citations{5 citations counted in INSPIRE as of 23 Apr 2019}

\item
{\bf {Search for a light charged Higgs boson in the $H^{+}\to c\bar{s}$ channel 
at 13 TeV, in the CMS experiment at the LHC}}
  \\{}S.R. Dugad, G. Kole, G.B. Mohanty, A. Nayak, R.K. Verma
  \\{} CMS AN-18-061, CMS HIG-18-021
  \\{} https://cds.cern.ch/record/2699812

\item
{\bf {Search for an excited lepton in the lepton + fatjet final state at 13 TeV, 
in the CMS experiment at the LHC}}
  \\{}S.B. Beri, K. Hoepfner, S. Dutt, S. Thakur, R.K. Verma
  \\{} CMS AN-18-126
\end{enumerate}
}

\item {\textbf{Indirect contribution:}
%\section*{Indirect contribution}
As a policy of the CMS experiment, a person automatically qualifies to become an author of a paper
if he/she participates in the data taking process. The data taking is a very extensive
process which requires a huge man power to operate and monitor the CMS detector on a daily basis. Therefore, each published paper from the CMS experiment has around 2500 authors because each author has contributed in some aspect of the data taking. Every year each person has to do a \textit{service task} for 4 months to become an author of the papers published in that year. 

On the management front, people are assigned a convener post for handling the responsibility of the given task. The hierarchy is like this - around ten level-3 (L3) convenors report to their level-2 (L2) convenor and around ten level-2 (L2) convenors report to the level level-1 (L1) convenor assigned to them.

As part of my service task, I served as an L3 convener in the alignment calibration and database (AlCaDB) sub-group of the detector performance group (DPG). The detector alignment constants (a set of detector conditions) are derived on a daily or weekly basis to calibrate the simulated data so that it matches with the observed data. To assure this, a dedicated validation is performed for every new set of constants. As part of the validation process, many  plots have to be scrutinized to make sure that these constants do not affect the performance of the detector. Once these constants are validated, they are stored in the database for future usage. Below, I have provided a brief summary of my work as an L3 convenor. 


\begin{itemize}[leftmargin=*]
\item {\textbf{As an L3 convener (2016-18) for the \textit{workflow submission and 
management}:}
%\subsection{As an L3 convener (2016-18) for the \textit{workflow submission and 
I served as an L3 convener from December 2016 to December 2018 for the workflow 
submission and management with my co-convener Bajrang (TIFR, Mumbai) (from December 2016 to 
December 2017) and Pritam (IIT, Madras) (from December 2017 to 2018). We performed \dq{tag} 
validations by submitting workflows and monitoring the DQM (data quality 
monitoring) plots. A \dq{tag} is a C++ string which contains condition data of the 
CMS experiment. Whenever, there is any change in any sub-detector of the CMS a 
\dq{new tag} is created to accommodate the change. Our task was to validate whether the \dq{new tag}
is correct or not by looking various DQM plots. The DQM plots are created when 
we submit workflows corresponding to the \dq{new tag}. We worked on the submission  
and validation of 687 (672) workflows in 2017 (2018). The alignment conditions in 
these workflows were derived from various sub-groups such as tracker, ECAL, HCAL, 
and muon chambers of the CMS experiment. We interacted with around 100 people from 
different sub-groups during the validation process. In each year, I got the service task credit for
four months.

From June 2017 to June 2018, I also worked on another service task in the 
AlCaDB group with Amey. We worked on a software package called the LHCInfo. It 
fetches LHC \dq{Fill} information from an online database and stores them in the offline 
database. It was originally developed by Salvatore. Various 
new variables such as lumi per bunch crossings, \verb|lhcState|, \verb|lhcComments|, 
\verb|ctppsStatus|, \verb|lumiSection| were not present in the old package 
developed by Salvatore. Therefore, we upgraded this package so that it can fetch many 
new variables related to the LHC Fill. The LHCInfo O2O was developed, commissioned 
and deployed in early May 2018. Since its deployment, approximately 0.44\unit{M} payloads 
have been populated in the condition database. The payloads for each Fill 
(with stable beam) are being populated with lumi-section granularity (that is 
after every 23 seconds) in the database. With this, one can access the relevant 
information of any LHC Fill within the CMSSW jobs which helps in correlating the 
beam information with sub-detector needs. In 2018, I got 2 months of service task credit for this. 
}
\item {\textbf{An an L3 convener (2019-21) for the \textit{AlCa-TSG contact}:}
After my tenure as an L3 convener for \textit{workflow submission and management} 
ended, I am recommended for new position in the AlCaDB group as AlCa-TSG (trigger study group) 
contact. This term is from Jan 2019 to Jan 2021. Our (mine and Ashish) job is to
present AlCaDB interest in the TSG group and vice versa. Currently, we have been 
assigned three tasks. The first task is to include a lot of AlCa conditions directly
in the global tag for the offline data. The second task is to list Alca conditions 
consumed in the latest CMSSW which are needed at different steps such as GEN, SIM,
DIGI, HLT, and RECO. The third task is to automate update or switch off the trigger 
bits.
}
\end{itemize}
During the last three years of the service task in the AlCaDB group, I worked under the guidance and leadership of five outstanding L2 convenors - Giovanni Franzoni, Giacomo Govi, Arun Kumar, Luca Pernie, and Tongguang Cheng. 
}
\end{itemize}
