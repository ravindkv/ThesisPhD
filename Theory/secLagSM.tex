
\section{Group and symmetry}
\label{s:smGroup}
In mathematics, a set $(G, \times)$ is called a group with respect to the 
multiplicative ($\times$) operator if the following conditions are satisfied: 
\begin{enumerate}
\item there exist an identity ($I$) element in $G$ so that $I\times g = g\times I = 
g$ where g is an element of $G$, 
\item the elements of the group follow associative 
law that is $(g_1\times g_2)\times g_3$ = $g_1\times (g_2\times g_3)$, and 
\item there exist an inverse of every element of $G$ so that $g\times g^{-1} = g^{-1}\times g = I$. 
\end{enumerate}
The group $(G, \times)$ is called an abelian group if 
the elements also follow commutative law that is $g_1\times g_2 = g_2\times g_1$. An example of a group in physics is the continuous rotation group. In general,
all the elements of a group can be generated by the parameters and generators 
of the group. For Lie groups, an element is given by $e^{i\alpha_a\tau_a}$ where
$\alpha_a$ (a = 1, 2, .., n) are the parameters and $\tau_a$ are the generators. For the theory 
of the standard model, we need the following three groups.
\begin{itemize} [leftmargin=*]
				\item \textbf{U(1)}: The unitary $1\times 1$ matrices form the U(1) 
								group. An element of the U(1) is given by $e^{i\alpha}$ where 
								$\alpha$ is the parameter. The identity element is the 
								generator of U(1).
				\item \textbf{SU(2)}: The elements of special unitary (SU) group are 
								$2\times 2$ unitary matrices. The \dq{special} implies that the
								determinant of each matrix is 1. There are in total 3 generators
								(the three Pauli matrices). All the elements of the SU(2) can be 
								constructed using these generators ($\tau_a$) and a set of 
								parameters ($\alpha_a$). An element of the SU(2) is given by 
								$e^{i\alpha_a\tau_a}$, where a = 1, 2,3. Out of three generators,
								one is diagonal. In general, the element of the SU(2) group do not 
								commute. Therefore, SU(2) is a non-Abelian group.
				\item \textbf{SU(3)}: A set of $3\times 3$ unitary matrices form the
								SU(3) group. There are 8 generators (Gell-Mann matrices) of the
								SU(3), two out of eight are diagonal. An element of the SU(3) 
								is given by $e^{i\alpha_a\lambda_a}$ where a = 1, 2, ..8, and
								$\lambda_a$ are the Gell-Mann matrices. In general, for a SU(N), 
								there are $N^2 -1$ generators, out of which $N-1$ are diagonal. 
								The SU(3) group is also non-Abelian.
\end{itemize}

A physical system is said to be invariant if it does not change under the
transformation of a group. For example, a function $f(\psi)$ is invariant 
under SU(2) if $f(\psi^\prime) = f(\psi)$ where $\psi^\prime = e^{i\alpha_a\tau_a} \psi$. There could be a function of more than one variable such as 
$f(\psi,\phi)$ where one of the variables (say $\psi$) might transform under 
SU(2) however, the other variable may not transform at all under this group. In
this case also, the function $f(\psi,\phi)$ will be invariant under SU(2). 
Therefore it is important to know about the transformation law of every variable
of a function. One can also construct a function where one variable transforms 
under one group (say SU(2)) and the other variable transforms under a different 
group (say U(1)). In this case, the function is said to be invariant under 
the combined group SU(2)$\times$ U(1). However, the generators of the individual 
group must commute for the invariance under combined group.

Noether theorem \cite{Noether:1918zz} demands that there is a conserved charge 
with every global (the parameters of the group are space-time independent) symmetry. 
For example, invariance under rotations leads to conservation of angular momentum. 
The quantum numbers for U(1) and SU(2) are denoted by $Q$ (called charge) and $T$ 
(called the weak-isospin). The third component of weak-isospin is denoted by $T_3$. In 
the context of the standard model, where electro-weak lagrangian (refer Section~\ref{s:smLag}) 
is invariant under the SU(2)$_L\times $U(1) group, an other 
quantum number called the hypercharge ($Y = 2( Q - T_3)$) is assigned to every 
particle. The $L$ in the SU(2)$_L$ implies that only left-handed fermions 
transform under the SU(2) group. The right-handed fermions are singlets, that is, 
they don't transform under this group. The various quantum numbers
associated with all the fundamental particles are shown in Table~\ref{tab:smGroup}. 
From this table, it can be seen that vector bosons 
($W^+$, $W^-$, \rm{ and } $Z$) form weak-isospin triplet, left-handed fermions form weak-isospin 
doublet, the Higgs boson is also part of the weak-isospin doublet, and the rest of 
the other particles (right-handed quarks and photon) are weak-isospin singlet.
\begin{table}
\caption{The third component of the weak-isospin, hypercharge, and charge quantum 
				numbers for leptons, quarks, and bosons.}
\label{tab:smGroup}
\begin{center}
\begin{tabular}{c c c c c c c}
\hline
\hline
				\multicolumn{1}{c}{Fermion type} &\multicolumn{3}{c}{Generation}
				& \multicolumn{1}{c}{$T_3$} 
				&\multicolumn{1}{c}{Y}& \multicolumn{1}{ c}{ Q } \\
				& 1st & 2nd  & 3rd & & & \\
\hline
\hline
				Leptons &$\begin{pmatrix} \nu_e \\ e \end{pmatrix}_L$ & $\begin{pmatrix} \nu_{\mu} \\ \mu \end{pmatrix}_L$ & $\begin{pmatrix} \nu_{\tau} \\ \tau \end{pmatrix}_L$  & $\begin{pmatrix} \frac{1}{2} \\ $-$\frac{1}{2} \end{pmatrix}$ & -1 & $\begin{pmatrix} 0 \\ $-1$ \end{pmatrix}$\\ [0.4cm]
				&$e_R$  & $\mu_R$ & $\tau_R$ & 0 & -2 & -1\\ [0.2cm]
				Quarks &$\begin{pmatrix} u \\ d \end{pmatrix}_L$ & $\begin{pmatrix} c \\ s \end{pmatrix}_L$ & $\begin{pmatrix} t \\ b \end{pmatrix}_L$  & $\begin{pmatrix} \frac{1}{2} \\ $-$\frac{1}{2} \end{pmatrix}$ & $\frac{1}{3}$ & $\begin{pmatrix} \frac{2}{3} \\ $-$\frac{1}{3} \end{pmatrix}$\\ [0.4cm]
				& $u_R$ & $c_R$ & $t_R$ & 0 &$\frac{4}{3}$ & $\frac{2}{3}$\\ [0.2cm]
				&$d_R$ & $s_R$ & $b_R$  & 0 & -$\frac{2}{3}$ & -$\frac{1}{3}$\\[0.2cm]\hline
				$W^+$ & & & & + 1 & 0 & + 1 \\[0.2cm]
				$W^-$ & & & & - 1 & 0 & - 1 \\[0.2cm]
				$Z$ & & & & 0 & 0 & 0 \\[0.2cm]
				$\gamma$ & & & & 0 & 0 & 0 \\[0.2cm]
				$H$ & & & & -$\frac{1}{2}$ & 1 & 0 \\[0.2cm]
\hline
\end{tabular}
\end{center}
\end{table}


\section{The Lagrangian density}
\label{s:smLag}
The physics of all fundamental particles is described by the Standard 
Model. The starting point of all physical theories is the 
construction of a Lagrangian density. The equations of motion for 
different fields, such as Klein-Gordon equation for scalar field, Dirac
equation for spinor field, Maxwell's equations of electromagnetism, etc, 
can be derived from a Lagrangian density using the Euler-Lagrange equation. For the
sake of convenience, we will call \dq{Lagrangian density} as \dq{Lagrangian} only.
The Lagrangian has to satisfy some basic principles.
First, it has to follow some symmetry principle, that is, it should be invariant under 
the transformation of the prescribed symmetry group. 
Second, the Lagrangian has to be \dq{renormalizable}. One 
of the criteria of renormalization demands that each interaction term in the Lagrangian
should have a coupling with mass dimension greater than or equals to zero 
(the mass dimension of fermion field, boson field, space-time derivative, and mass is 
3/2, 1, 1, 1, respectively). The advantage of Lagrangian formulation is that the Feynman 
rules can be obtained just by looking at the mathematical structure of individual terms, 
for example, the propagator is determined by the quadratic terms, and the vertex factor is 
determined by the interaction terms. The standard model (SM) is a unified theory incorporating 
electromagnetic, weak, and strong interactions. The SM Lagrangian is invariant 
under the SU(3)$_C\times$ SU(2)$_L\times$ U(1)$_Y$ group and written as
\begin{equation}
	\mathcal{L}_{\text{SM}} =  \mathcal{L}_\text{EW} + \mathcal{L}_\text{QCD} +
	\mathcal{L}_\text{H} + \mathcal{L}_\text{Yukawa}
  \label{eq:smLag}
\end{equation}
where individual terms are from electroweak, quantum chromodynamics, Higgs, and
Yukawa sectors. A detailed description of each individual term is given below.

\begin{itemize} [leftmargin=*]
\item \textbf{$\mathcal{L}_\text{EW}$}: The electro-weak (EW) Lagrangian
				follows SU(2)$_L\times$ U(1)$_Y$ symmetry and incorporates the interaction 
				between leptons (and quarks) and electro-weak gauge bosons along with the respective kinetic 
				energy terms. It is written as
				\begin{equation}
								\mathcal{L}_\text{EW} = \bar L \gamma^\mu \left(i\partial_\mu 
								- g^\prime \tfrac12 Y B_\mu - g \tfrac12 \tau_\text{a} 
								W_\mu^{\text{a}}\right)L + \bar R \gamma^\mu \left(i\partial_\mu 
								- g^\prime \tfrac12 Y B_\mu \right)R - \tfrac{1}{4} W_a^{\mu\nu} 
								W_{\mu\nu}^a - \tfrac{1}{4} B^{\mu\nu} B_{\mu\nu},
				\label{eq:LagEW}
				\end{equation}
				where $L$ ($\bar{L} = L^\dagger 
				\gamma^0$) is the doublet of left-handed leptons ($e_L, \nu_{e_L}; \mu_L, 
				\nu_{\mu_L}; \tau_L, \nu_{\tau_L}$) as shown in the first row of Table 
				\ref{tab:smGroup}, $R$ is the singlet of 
				right-handed lepton (only $e_R, \mu_R, \tau_R$) as shown in the second 
				row of Table~\ref{tab:smGroup}, the $\gamma^\mu$ are the Dirac matrices, $\tau_a$ are Pauli matrices, 
				$g^\prime$ and $g$ are coupling strength of the U(1) and SU(2) group, 
				respectively. In Equation (\ref{eq:LagEW}), we have not explicitly
shown the terms corresponding to quarks. The $B_\mu$ is the U(1) gauge field, 
				$W^a$ are the SU(2) gauge fields, $B_{\mu\nu} = 
				\partial_\mu 
				B_\nu - \partial_\nu B_\mu$ and $W_{\mu\nu}^a = \partial_\mu W_\nu^a - 
				\partial_\nu W_\mu^a + g \epsilon^{abc}W_\mu^bW_\nu^c$ are the 
				corresponding field strength tensors. Here $\epsilon^{abc}$ is totally 
antisymmetric Levi-Civita symbol. The $Y$ in Equation (\ref{eq:LagEW}) is the 
				hypercharge of the U(1) group. The summation in Equation (\ref{eq:LagEW}) 
				runs over three spaces; three generation of lepton, three generators of 
				the SU(2)(a = 1, 2, 3), and the 4 Lorentz indices ($\mu$ = 0, 1,2 ,3). 
				In Equation (\ref{eq:LagEW}), there is no coupling of $W^a$ with the 
				right-handed lepton, as they interact only with the left-handed 
				lepton. Under
				the SU(2)$_L\times$ U(1)$_Y$ group, the field $L, ~R, ~B_\mu$, and $W_\mu^a$ transform as $L^\prime = 
				e^{i\alpha_a(x)T^a + i\beta(x)Y}L, ~R^\prime = e^{i\beta(x)Y}R, 
				~B_\mu^\prime = B_\mu + \frac{1}{e}\partial_\mu\alpha(x), ~W_\mu^{\prime a}
				= W_\mu^a - \frac{1}{g}\partial_\mu\alpha_a(x) - f_{abc}\alpha_b(x) 
				W^c_\mu$, where $\alpha (x)$ and $\beta (x)$ are the local gauge parameters
				of the SU(2) and U(1) group, respectively. The Lagrangian $\mathcal{L}_\text{EW}$ is 
				invariant under these transformations.

\item \textbf{$\mathcal{L}_\text{QCD}$}: The Lagrangian for the quantum
chromodynamics involves the interactions of quarks and gluons and 
respects SU(3) symmetry. Unlike the lepton, the quarks have an additional 
color quantum number called \dq{color}. The Lagrangian is given as
\begin{equation}
\mathcal{L}_\text{QCD} = \bar Q_{i} \left( i\gamma^\mu(
\partial_\mu\delta_{ij} - i g_s G_\mu^a T^a_{ij})\right) Q_j
- \frac{1}{4} G^a_{\mu\nu} G^{\mu\nu}_a,
\label{eq:LagQCD}
\end{equation}
where the $i$ in $Q_i$ is the color index for $R$, $G$, and $B$. The $Q$ is
the Dirac spinor for each quark, $g_s$ is the strength
of strong coupling, $T^a$ are 8 (a = 1, 2, .., 8) generators of the SU(3)
also called Gell-Mann matrices, $G^a$ are the gauge fields of the strong 
interaction, $G_{\mu\nu}^a = \partial_\mu G_\nu^a - \partial_\nu G_\mu^a + g 
f^{abc}G_\mu^bG_\nu^c$ are the gluon field strength tensors, and $f^{abc}$ 
are the structure constants of SU(3). There are five summations involved
in Equation (\ref{eq:LagQCD}): quark family (Q = u, d, c, s, t, b), 
chirality of quarks, color (i = 1, 2, 3), a = 1, 2, .., 8, and
Lorentz ($\mu$ = 0, 1, 3, 4). Under the SU(3), the two fields transform as 
$Q^\prime = e^{i\alpha_a(x)T^a }Q, ~G_\mu^{\prime a} = G_\mu^a - 
\frac{1}{g}\partial_\mu\alpha_a(x) - f_{abc}\alpha_b(x) G^c_\mu$. The 
Lagrangian $\mathcal{L}_\text{QCD}$ is invariant under these transformations.

\item \textbf{$\mathcal{L}_\text{H}$}: The Higgs sector of the SM incorporates
				interaction of Higgs fields with gauge bosons of SU(2) and U(1) groups
				as well as a potential energy term for Higgs. The Higgs field is part 
				of a complex scalar doublet of the SU(2) group. The Lagrangian density is 
				given as
				\begin{equation}
				\mathcal{L}_\text{H} = \left|\left(\partial_\mu + \frac{i}{2} \left( 
								g^\prime Y B_\mu + g \tau_a W_\mu^a \right)\right)\varphi
								\right|^2 - V(\varphi), 
				\label{eq:LagHiggs}
				\end{equation}
				where the Higgs doublet is defined as
				\begin{equation}
				\varphi = \frac{1}{\sqrt 2} \left(\begin{array}{c}\varphi^+ \\ 
				\varphi^0\end{array}\right),
				\end{equation}
				with $\varphi^+$ and $\varphi^0$ having electrical charge +1 and 0, 
respectively. 
				The hypercharge of both fields is 1. The potential term for the Higgs 
				field $V(\varphi)$ is given in Section~\ref{s:smMass}. Under the SU(2)
				group, the Higgs doublet transforms as $\varphi^\prime = e^{i\alpha_a(x)
				\tau^a}\varphi$. The transformation law for the other two fields of 
				Equation (\ref{eq:LagHiggs}) is described in $\mathcal{L}_\text{EW}$. 
				The mass to the vector bosons is generated in $\mathcal{L}_\text{H}$ as
				discussed in Section~\ref{s:smMass}.

\item \textbf{$\mathcal{L}_\text{Yukawa}$}: The interactions of fermions (quarks and leptons) with 
Higgs boson are described by the Yukawa Lagrangian. It is written as 
\begin{equation}
\mathcal{L}_\text{Yukawa} =  G_l\bar L \varphi R + G_d \bar{q_L} \varphi d_R +G_u \bar{q_L} \varphi_c u_R + h.c.,
\end{equation}
where $G_l$ is the coupling of Higgs to lepton, $G_d$ ($G_u$) is the coupling
of Higgs to up (down) quark, and $\varphi_c = -i\tau_2\varphi^{\text{*}}$. 
\end{itemize}
