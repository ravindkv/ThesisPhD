%\documentclass[]{revtex4-1}
\documentclass[a4paper,10pt]{article}
\usepackage{amsmath,amssymb,amsfonts,amsthm,tikz,caption,subcaption}
\usepackage{tikz}
\usepackage{multicol} 
\usetikzlibrary{positioning,arrows}
\usetikzlibrary{decorations.pathmorphing}
\usetikzlibrary{decorations.markings}

\begin{document}

% Define lines according to your own choice
   \tikzset{
particle1/.style={thin,draw=black, postaction={decorate},
    decoration={markings,mark=at position .6 with {\arrow[black]{angle 45}}}},
particle2/.style={thin,draw=black, postaction={decorate},
    decoration={markings,mark=at position .6 with {\arrow[black]{triangle 45}}}},
particle3/.style={dashed,draw=black, postaction={decorate},
    decoration={markings,mark=at position .6 with {\arrow[black]{angle 45}}}},
gluon/.style={decorate, draw=black,
    decoration={coil,amplitude=4pt, segment length=5pt}}
}
 \tikzset{
phot/.style={decorate, draw=black,
    decoration={snake,amplitude=4pt, segment length=5pt}}    
}


\begin{figure*}[]
%\centering
\begin{multicols}{2}
%--------------------------------------
%            e3       e5
% e1	     |_______/
%	\________/a3  a5 \ e6
%   /a1	   a2\_______/e7	
% e2         |a4   a6\e8
%            e4    
%--------------------------------------
\begin{subfigure}[b]{1.0\linewidth}
% ============= gg fusion to Higgs ========================
\begin{tikzpicture}[node distance=1cm and 1.5cm]
\coordinate[label=left:$\bar{q}$] (e1);
\coordinate[below right=1.00cm of e1] (aux1);
\coordinate[below left=1.00cm of aux1,label=left:$q$] (e2);
\coordinate[right=1.25cm of aux1] (aux2);

\coordinate[above right=1.25cm of aux2] (aux3);
\coordinate[below right=1.25cm of aux2] (aux4);

\coordinate[above =1.00cm of aux3,label=above:$b$] (e3);
\coordinate[right=1.25cm of aux3] (aux5);

\coordinate[above right=1.0cm of aux5,label=right:$q^\prime$] (e5);
\coordinate[below right=1.0cm of aux5,label=right:$\bar{q}$] (e6);

\coordinate[right=1.25cm of aux4] (aux6);
\coordinate[below =1.0cm of aux4,label=below:$\bar{b}$] (e4);
\coordinate[above right=1.0cm of aux6,label=right:$\bar{\nu_l}$] (e7);
\coordinate[below right=1.0cm of aux6,label=right:$l^-$] (e8);

% Join all the points with preferred choice of lines
\draw[particle1] (e1) -- (aux1);
\draw[particle1] (aux1) -- (e2);
\draw[gluon] (aux1) -- node[label=above:$g$]{} (aux2);
\draw[particle1] (aux2) -- node[label=above:$t$]{} (aux3);
\draw[phot] (aux3) -- node[label=below:$W^+$]{} (aux5);
\draw[particle1] (aux3) -- (e3);
\draw[particle1] (aux5) -- (e5);
\draw[particle1] (e6) -- (aux5);

\draw[particle1] (aux4) -- node[label=below:$\bar{t}$]{} (aux2);
\draw[phot] (aux4) -- node[label=above:$W^-$]{}  (aux6);
\draw[particle1] (e4) -- (aux4);
\draw[particle1] (e7) -- (aux6);
\draw[particle1] (aux6) -- (e8);
\end{tikzpicture}
\caption{{\bf{Background}}: s-channel quark-quark\\scattering.} \label{fig:gg}
\end{subfigure}
%\vspace{20pt}

%\columnbreak
\begin{subfigure}[b]{1.5\linewidth}
% ============= gg fusion to Higgs ========================
\begin{tikzpicture}[node distance=1cm and 1.5cm]
\coordinate[label=left:$g$] (e1);
\coordinate[below right=1.00cm of e1] (aux1);
\coordinate[below left=1.00cm of aux1,label=left:$g$] (e2);
\coordinate[right=1.25cm of aux1] (aux2);

\coordinate[above right=1.25cm of aux2] (aux3);
\coordinate[below right=1.25cm of aux2] (aux4);

\coordinate[above =1.00cm of aux3,label=above:$b$] (e3);
\coordinate[right=1.25cm of aux3] (aux5);

\coordinate[above right=1.0cm of aux5,label=right:$q^\prime$] (e5);
\coordinate[below right=1.0cm of aux5,label=right:$\bar{q}$] (e6);

\coordinate[right=1.25cm of aux4] (aux6);
\coordinate[below =1.0cm of aux4,label=below:$\bar{b}$] (e4);
\coordinate[above right=1.0cm of aux6,label=right:$\bar{\nu_l}$] (e7);
\coordinate[below right=1.0cm of aux6,label=right:$l^-$] (e8);

% Join all the points with preferred choice of lines
\draw[gluon] (e1) -- (aux1);
\draw[gluon] (aux1) -- (e2);
\draw[gluon] (aux1) -- node[label=above:$g$]{} (aux2);
\draw[particle1] (aux2) -- node[label=above:$t$]{} (aux3);
\draw[phot] (aux3) -- node[label=below:$W^+$]{} (aux5);
\draw[particle1] (aux3) -- (e3);
\draw[particle1] (aux5) -- (e5);
\draw[particle1] (e6) -- (aux5);

\draw[particle1] (aux4) -- node[label=below:$\bar{t}$]{} (aux2);
\draw[phot] (aux4) -- node[label=above:$W^-$]{}  (aux6);
\draw[particle1] (e4) -- (aux4);
\draw[particle1] (e7) -- (aux6);
\draw[particle1] (aux6) -- (e8);
\end{tikzpicture}
\caption{{\bf{Background}}: s-channel gluon-gluon\\scattering.}
\label{fig:gg}
\end{subfigure}

%--------------------------------------
%			  e5	
%          a5/__e6
% e1\ ______/    
%    |a1  a3\ 
%    |      e3
%    |      e4
%    |______/	 	
%   / a2  a4\__e7
% e2      a6 \            
%            e8
%--------------------------------------

\begin{subfigure}[b]{1.5\linewidth}
% ============= gg fusion to Higgs ========================
\begin{tikzpicture}[node distance=1cm and 1.5cm]
\coordinate[label=left:$g$] (e1);
\coordinate[below right=1.00cm of e1] (aux1);
\coordinate[below=2.00cm of aux1] (aux2);
\coordinate[below left =1.00cm of aux2,label=left:$g$] (e2);

\coordinate[right=1.25cm of aux1] (aux3);
\coordinate[right=1.25cm of aux2] (aux4);

\coordinate[above right=1.25cm of aux3] (aux5);
\coordinate[below right =1.00cm of aux3,label=right:$b$] (e3);


\coordinate[above =1.0cm of aux5,label=right:$q^\prime$] (e5);
\coordinate[right=1.0cm of aux5,label=right:$\bar{q}$] (e6);

\coordinate[above right =1.0cm of aux4,label=right:$\bar{b}$] (e4);
\coordinate[below right=1.25cm of aux4] (aux6);
\coordinate[right=1.0cm of aux6,label=right:$\bar{\nu_l}$] (e7);
\coordinate[below right=1.0cm of aux6,label=right:$l^-$] (e8);

% Join all the points with preferred choice of lines
\draw[gluon] (e1) -- (aux1);
\draw[gluon] (e2) -- (aux2);
\draw[particle1] (aux1) -- node[label=above:$t$]{} (aux3);
\draw[particle1] (aux1) -- node[label=left:$t$]{} (aux2);
\draw[particle1] (aux4) -- node[label=above:$\bar{t}$]{} (aux2);
\draw[phot] (aux3) -- node[label=below right:$W^+$]{} (aux5);
\draw[particle1] (aux3) -- (e3);
\draw[particle1] (aux5) -- (e5);
\draw[particle1] (e6) -- (aux5);

\draw[phot] (aux4) -- node[label=above right:$W^-$]{}  (aux6);
\draw[particle1] (e4) -- (aux4);
\draw[particle1] (e7) -- (aux6);
\draw[particle1] (aux6) -- (e8);
\end{tikzpicture}
\caption{{\bf{Background}}: t-channel gluon-gluon\\scattering.}
\label{fig:gg}
\end{subfigure}
%--------------------------------------

\begin{subfigure}[b]{1.0\linewidth}
% ============= gg fusion to Higgs ========================
\begin{tikzpicture}[node distance=1cm and 1.5cm]
\coordinate[label=left:$\bar{q}$] (e1);
\coordinate[below right=1.00cm of e1] (aux1);
\coordinate[below left=1.00cm of aux1,label=left:$q$] (e2);
\coordinate[right=1.25cm of aux1] (aux2);

\coordinate[above right=1.25cm of aux2] (aux3);
\coordinate[below right=1.25cm of aux2] (aux4);

\coordinate[above =1.00cm of aux3,label=above:$b$] (e3);
\coordinate[right=1.25cm of aux3] (aux5);

\coordinate[above right=1.0cm of aux5,label=right:$c$] (e5);
\coordinate[below right=1.0cm of aux5,label=right:$\bar{s}$] (e6);

\coordinate[right=1.25cm of aux4] (aux6);
\coordinate[below =1.0cm of aux4,label=below:$\bar{b}$] (e4);
\coordinate[above right=1.0cm of aux6,label=right:$\bar{\nu_l}$] (e7);
\coordinate[below right=1.0cm of aux6,label=right:$l^-$] (e8);

% Join all the points with preferred choice of lines
\draw[particle1] (e1) -- (aux1);
\draw[particle1] (aux1) -- (e2);
\draw[gluon] (aux1) -- node[label=above:$g$]{} (aux2);
\draw[particle1] (aux2) -- node[label=above:$t$]{} (aux3);
\draw[particle3] (aux3) -- node[label=below:$H^+$]{} (aux5);
\draw[particle1] (aux3) -- (e3);
\draw[particle1] (aux5) -- (e5);
\draw[particle1] (e6) -- (aux5);

\draw[particle1] (aux4) -- node[label=below:$\bar{t}$]{} (aux2);
\draw[phot] (aux4) -- node[label=above:$W^-$]{}  (aux6);
\draw[particle1] (e4) -- (aux4);
\draw[particle1] (e7) -- (aux6);
\draw[particle1] (aux6) -- (e8);
\end{tikzpicture}
\caption{{\bf{Signal}}: s-channel quark-quark\\scattering.}
\label{fig:gg}
\end{subfigure}
%\vspace{20pt}

%\columnbreak
\begin{subfigure}[b]{1.5\linewidth}
% ============= gg fusion to Higgs ========================
\begin{tikzpicture}[node distance=1cm and 1.5cm]
\coordinate[label=left:$g$] (e1);
\coordinate[below right=1.00cm of e1] (aux1);
\coordinate[below left=1.00cm of aux1,label=left:$g$] (e2);
\coordinate[right=1.25cm of aux1] (aux2);

\coordinate[above right=1.25cm of aux2] (aux3);
\coordinate[below right=1.25cm of aux2] (aux4);

\coordinate[above =1.00cm of aux3,label=above:$b$] (e3);
\coordinate[right=1.25cm of aux3] (aux5);

\coordinate[above right=1.0cm of aux5,label=right:$c$] (e5);
\coordinate[below right=1.0cm of aux5,label=right:$\bar{s}$] (e6);

\coordinate[right=1.25cm of aux4] (aux6);
\coordinate[below =1.0cm of aux4,label=below:$\bar{b}$] (e4);
\coordinate[above right=1.0cm of aux6,label=right:$\bar{\nu_l}$] (e7);
\coordinate[below right=1.0cm of aux6,label=right:$l^-$] (e8);

% Join all the points with preferred choice of lines
\draw[gluon] (e1) -- (aux1);
\draw[gluon] (aux1) -- (e2);
\draw[gluon] (aux1) -- node[label=above:$g$]{} (aux2);
\draw[particle1] (aux2) -- node[label=above:$t$]{} (aux3);
\draw[particle3] (aux3) -- node[label=below:$H^+$]{} (aux5);
\draw[particle1] (aux3) -- (e3);
\draw[particle1] (aux5) -- (e5);
\draw[particle1] (e6) -- (aux5);

\draw[particle1] (aux4) -- node[label=below:$\bar{t}$]{} (aux2);
\draw[phot] (aux4) -- node[label=above:$W^-$]{}  (aux6);
\draw[particle1] (e4) -- (aux4);
\draw[particle1] (e7) -- (aux6);
\draw[particle1] (aux6) -- (e8);
\end{tikzpicture}
\caption{{\bf{Signal}}: s-channel gluon-gluon\\scattering.}
\label{fig:gg}
\end{subfigure}


\begin{subfigure}[b]{1.5\linewidth}
% ============= gg fusion to Higgs ========================
\begin{tikzpicture}[node distance=1cm and 1.5cm]
\coordinate[label=left:$g$] (e1);
\coordinate[below right=1.00cm of e1] (aux1);
\coordinate[below=2.00cm of aux1] (aux2);
\coordinate[below left =1.00cm of aux2,label=left:$g$] (e2);

\coordinate[right=1.25cm of aux1] (aux3);
\coordinate[right=1.25cm of aux2] (aux4);

\coordinate[above right=1.25cm of aux3] (aux5);
\coordinate[below right =1.00cm of aux3,label=right:$b$] (e3);


\coordinate[above =1.0cm of aux5,label=right:$c$] (e5);
\coordinate[right=1.0cm of aux5,label=right:$\bar{s}$] (e6);

\coordinate[above right =1.0cm of aux4,label=right:$\bar{b}$] (e4);
\coordinate[below right=1.25cm of aux4] (aux6);
\coordinate[right=1.0cm of aux6,label=right:$\bar{\nu_l}$] (e7);
\coordinate[below right=1.0cm of aux6,label=right:$l^-$] (e8);

% Join all the points with preferred choice of lines
\draw[gluon] (e1) -- (aux1);
\draw[gluon] (e2) -- (aux2);
\draw[particle1] (aux1) -- node[label=above:$t$]{} (aux3);
\draw[particle1] (aux1) -- node[label=left:$t$]{} (aux2);
\draw[particle1] (aux4) -- node[label=above:$\bar{t}$]{} (aux2);
\draw[particle3] (aux3) -- node[label=below right:$H^+$]{} (aux5);
\draw[particle1] (aux3) -- (e3);
\draw[particle1] (aux5) -- (e5);
\draw[particle1] (e6) -- (aux5);

\draw[phot] (aux4) -- node[label=above right:$W^-$]{}  (aux6);
\draw[particle1] (e4) -- (aux4);
\draw[particle1] (e7) -- (aux6);
\draw[particle1] (aux6) -- (e8);
\end{tikzpicture}
\caption{{\bf{Signal}}: t-channel gluon-gluon\\scattering.}
\label{fig:gg}
\end{subfigure}
\end{multicols}
\end{figure*}
\end{document}